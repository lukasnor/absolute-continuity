Sei $\gamma:[0,a]\to M$ eine Geodätische. \pause Ein Punkt $\gamma(t_0)$, \pause $t_0\in (0,a]$, \pause heißt \emph{konjugiert} \pause zu $\gamma(0)$ \pause entlang $\gamma$, \pause falls ein Jacobifeld $J\neq 0$ entlang $\gamma$ existiert \pause mit $J(0)=0 = J(t_0)$.\pause\\
Die Menge $C(p)$ der (ersten) zu $p\in M$ konjugierten Punkte \pause entlang aller Geodätischen ausgehend von $p$ \pause wird \emph{conjugate locus} von $p$ genannt.