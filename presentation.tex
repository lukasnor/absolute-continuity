\documentclass[ngerman
			,handout
			]{beamer}
\usepackage[utf8]{inputenc}
\usepackage[T1]{fontenc}
\usepackage[ngerman]{babel}
\usetheme[numbering=none,nofirafonts]{focus}
\usepackage{newpxtext,newpxmath}
%\usetheme{Madrid}
\usepackage{nameref}
\usepackage{csquotes}
\usepackage{mathtools}
\usepackage{verbatim}

\title{Der Hauptsatz der Differential- und Integralrechnung für absolut stetige Funktionen}
%\subtitle{Seminar Klassische Sätze der Differentialgeometrie (WS 21/22)}
\author{Lukas Norkowski}
\date{30. Mai 2022} 

\newcommand{\RR}{\mathbb{R}}
\newcommand{\ZZ}{\mathbb{Z}}
\newcommand{\DD}{\mathcal{D}}
\newcommand{\MM}{\mathcal{M}}
\newcommand{\KK}{\mathcal{K}}
\newcommand{\HH}{\mathcal{H}}

\renewcommand{\d}{\mathrm{d}}

\hyphenation{Ex-po-nen-tial-ab-bil-dung}

%\usefonttheme[onlymath]{serif}
\usefonttheme{serif}
\renewcommand<>{\emph}[1]{{\only#2{\em}#1}}
\theoremstyle{definition}
\newtheorem*{theorem*}{Theorem}
\newtheorem{satz}[theorem]{Satz}
\newtheorem{hsatz}[theorem]{Hilfssatz}
\newtheorem*{remark}{\translate{Remark}}
\newtheorem*{reminder}{\translate{Reminder}}
\newtheorem{cor}[theorem]{Corollary}
\newtheorem*{cor*}{Corollary}
\newtheorem{proposition}[theorem]{\translate{Proposition}}


\newtheorem{memo}{Memo}
\newenvironment<>{memo}{
  \setbeamercolor{block title alerted}{fg=black, bg=yellow}
  \setbeamercolor{block body alerted}{bg=yellow!25}
  \begin{alertblock}{Memo}}{\end{alertblock}}

\begin{document}

\begin{frame}
	\titlepage
\end{frame}	

\begin{frame}{Gliederung}
\tableofcontents
\end{frame}


\section{Einleitung und Motivation}
\label{Einleitung und Motivation}

\begin{frame}{Bereits bekannt}
	\begin{theorem*}[Fundamentalsatz für stetig differentierbare Funktionen]
		\pause
		Es sei $f\colon [a,b]\to \RR$ eine differentierbare Funktion mit stetiger Ableitung. Dann gilt\pause
\begin{equation*}
    f(x)-f(a) = \int_a^x f'(t)\d t,\quad x\in[a,b].
\end{equation*}
	\end{theorem*}
\end{frame}

\begin{frame}{Fragen}
	\begin{enumerate}
		\item Foo
	\end{enumerate}
\end{frame}

\begin{frame}{\enquote{Gegenbeispiele}}
	
\end{frame}



\section{Das Lebesguemaß}
\label{Das Lebesguemaß}


\begin{frame}{Das Lebesguemaß}
	Es sei $(\RR^d, \mathcal{B}_d, \mu_d)$ der Borel-Maßraum und die Menge aller Teilmengen von $\mu_d$-Nullmengen
	\begin{equation*}
		\mathcal{N}_d\coloneqq \{N\subseteq\RR^d\colon \text{Es existiert } M\in\mathcal{B}_d \text{ mit } N\subseteq M \text{ und }\mu_d(M)=0\}.
	\end{equation*}
	\begin{definition}
	\begin{itemize}
		\item Die Menge $\mathcal{L}_d \coloneqq \{B\cup N\colon B\in\mathcal{B}_d, N\in\mathcal{N}_d\}$ wird \term{Lebesguesche} $\sigma$-Algebra genannt.
		\item Die Abbildung $\lambda_d\colon\mathcal{L}_d\to[0, \infty]$, definiert durch $\lambda_d(L)\coloneqq\mu_d(B)$ für $L=B\cup N\in\mathcal{L}_d$, $B\in\mathcal{B}_d, N\in\mathcal{N}_d$, wird \term{Lebesguemaß} genannt.
	\end{itemize}
	\end{definition}
\end{frame}

\begin{frame}{Es geht mit rechten Dingen zu.}

\begin{memo}
	$\mathcal{L}_d \coloneqq \{B\cup N\colon B\in\mathcal{B}_d, N\in\mathcal{N}_d\}$,\\
	$\lambda_d\colon\mathcal{L}_d\to[0, \infty]$, $\lambda_d(L)=\mu_d(B)$ für $L=B\cup N\in\mathcal{L}_d$
\end{memo}
\begin{satz}
\begin{enumerate}
	\item Die Menge $\mathcal{L}_d$ ist eine $\sigma$-Algebra.
	\item Die Abbildung $\lambda_d$ ist wohldefiniert und ein Maß auf $(\RR^d, \mathcal{L}_d)$.
	\item $(\RR^d, \mathcal{L}_d, \lambda_d)$ ist \term{vollständig}, d.h. jede Teilmenge einer $\lambda_d$-Nullmenge ist wieder eine $\lambda_d$-Nullmenge und damit insbesondere messbar.
\end{enumerate}
\end{satz}
\begin{satz}
	Eine Menge $M\subseteq \RR^d$ ist genau dann in $\mathcal{L}_d$ enthalten, wenn für jedes $\varepsilon>0$ eine offene Menge $O\subseteq\RR^d$ und eine abgeschlossene Menge $A\subseteq\RR^d$ mit $A\subseteq M\subseteq O$ und $\lambda_d(O\setminus A)<\varepsilon$ existieren.
\end{satz}
\end{frame}


\section{Absolut stetige Funktionen}
\label{Absolut stetige Funktionen}

\begin{frame}{Absolute Stetigkeit von Maßen}
	Es sei $(X, \mathcal{A})$ ein Messraum und $\lambda$ und $\mu$ zwei Maße auf $(X,\mathcal{A})$.
	\begin{definition}
		$\lambda$ ist absolut stetig bezüglich $\mu$ ($\lambda\ll\mu$), falls gilt:
		\begin{equation*}
			\forall A\in\mathcal{A}\colon\:\mu(A)=0 \implies \lambda(A)=0.
		\end{equation*}
	\end{definition}
	\begin{hsatz}
		Sind $(X,\mathcal{A})$, $\lambda$ und $\mu$ wie oben und zusätzlich $\lambda$ \emph{endlich} (oder komplex), so gilt:
		\begin{equation*}
			\lambda\ll\mu \Longleftrightarrow \forall\varepsilon>0\:\exists\delta>0\:\forall A\in\mathcal{A}\colon \mu(A)<\delta \Rightarrow \abs{\lambda}(A)<\varepsilon.
		\end{equation*}
	\end{hsatz}
\end{frame}

\begin{frame}{Motivation}
	Ist $f'\in L^1[a,b]$, so definiert $\d \lambda =f' \d\mu$ ein Maß $\lambda$ auf $[a,b]$. $\lambda$ endlich bzw. komplex und daher stark absolut stetig nach vorigem Satz. Also gibt es zu jedem $\varepsilon>0$ ein $\delta>0$, sodass für jedes $n\in\NN$ und für alle disjunkten Teilintervalle $(a_1,b_1),\dots,(a_n,b_n)$ von $[a,b]$ gilt:
	\begin{equation*}
		\sum_{i=1}^n b_i-a_i <\delta \implies \sum_{i=1}^n \abs{\int_{a_i}^{b_i} f'(t)\d t }<\varepsilon.
	\end{equation*}
	\begin{definition}
		Eine Funktion $f\colon[a,b]\to \CC$ heißt \term{absolut stetig} auf $[a,b]$, falls es zu jedem $\varepsilon>0$ ein $\delta>0$ gibt, so dass für jedes $n\in\NN$ und alle disjunkten Teilintervalle $(a_1,b_1),\dots,(a_n,b_n)$ in $[a,b]$ gilt: 
\begin{equation*}
    \sum_{i=1}^n b_i - a_i <\delta \quad \implies \quad \sum_{i=1}^n \abs{f(b_i) - f(a_i)}<\varepsilon.
\end{equation*}
	\end{definition}
\end{frame}

\begin{frame}{Bemerkungen}
	\begin{remark}
	\begin{itemize}
		\item Jede absolut stetige Funktion ist insbesondere stetig ($n=1$).
		\item 
	\end{itemize}
	\end{remark}
	
\end{frame}

\begin{frame}{Charakterisierung von absoluter Stetigkeit}
	\begin{satz}
		Sei $f:[a,b]\to\RR$ stetig und monoton wachsend. Dann sind äquivalent:
\begin{enumerate}
\item $f$ ist auf $[a,b]$ absolut stetig,
\item $f$ bildet Nullmengen auf Nullmengen ab,
\item $f$ ist fast überall differentierbar auf $[a,b]$, $f'\in L^1[a,b]$ und
\begin{equation*}
    f(x)-f(a)=\int_a^x f'(t) \d t, \quad x\in[a,b].
\end{equation*}
\end{enumerate}
	\end{satz}
	\begin{alertblock}{Bemerkung}
		Die Cantorfunktion bildet eine Nullmenge auf eine Menge mit Maß $1$ ab!
	\end{alertblock}
\end{frame}

\section{Der Hauptsatz}
\label{Der Hauptsatz}

\begin{frame}{Der Hauptsatz}
	\begin{theorem*}
		Sei $f\colon[a,b]\to \CC$ absolut stetig. \pause Dann ist $f$ fast überall differentierbar auf $[a,b]$, \pause $f'\in L^1[a,b]$, \pause und
\begin{equation*}
    f(x) - f(a) = \int_a^x f'(t)\d t, \quad x\in[a,b].
\end{equation*}		
	\end{theorem*}
\end{frame}

\begin{frame}{Totalvariation}
	\begin{definition}
		Sei $f\colon [a,b]\to\CC$ eine Funktion. \pause Dann bezeichnet die \term{Totalvariationsfunktion} von $f$ die Funktion $F\colon[a,b]\to [0,\infty]$, definiert durch \pause
\begin{equation*}
    F(x)\coloneqq \sup \sum_{i=1}^N \abs{f(t_i) - f(t_{i-1})}, \quad x\in[a,b],
\end{equation*}
\pause
wobei das Supremum über alle $N\in\NN$ \pause und alle $a =t_0<t_1<\dots<t_n = x$ genommen wird. \pause

$f$ ist von \term{beschränkter Variation}, falls $F(b)<\infty$. \pause $F(b)$ bezeichnet die \term{Totalvariation} von $f$.
	\end{definition}
	\begin{remark}
		Ist $f\colon[a,b]\to\RR$ ein stetiger Weg. Dann hat $f$ genau dann endliche Länge $L$, wenn $f$ von beschränkter Variation ist.
	\end{remark}
\end{frame}

\begin{frame}{Satz über Totalvariation}
	\begin{satz}
		Sei $f:[a,b]\to\CC$ absolut stetig. Dann ist $f$ von beschränkter Variation. Ist $F$ die Totalvariationsfunktion von $f$, so sind die Funktionen $F, F+f$ und $F-f$ alle monoton wachsend und absolut stetig.
	\end{satz}
\end{frame}

\begin{frame}{Fundamentalsatz für absolut stetige Funktionen}
	\begin{theorem}
		Sei $f\colon[a,b]\to \CC$ absolut stetig. \pause Dann ist $f$ fast überall differentierbar auf $[a,b]$, \pause $f'\in L^1[a,b]$, \pause und
\begin{equation*}
    f(x) - f(a) = \int_a^x f'(t)\d t, \quad x\in[a,b].
\end{equation*}
	\end{theorem}
\end{frame}

\begin{frame}{Quellen}
	\begin{enumerate}
		\item \emph{Real And Complex Analysis}. Walter Rudin. Third Edition. McGraw-Hill International Editions. 1987.
		\item \emph{Analysis 3}. Michael Plum. Jonathan Wunderlich. \emph{3. Übungsblatt}. WS 18/19.
	\end{enumerate}
\end{frame}

\end{document}