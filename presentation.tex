\documentclass[ngerman
			,handout
			]{beamer}
\usepackage[utf8]{inputenc}
\usepackage[T1]{fontenc}
\usepackage[ngerman]{babel}
\usetheme[numbering=none]{focus}
\usepackage{nameref}
\usepackage{csquotes}
\usepackage{mathtools}
\usepackage{verbatim}

\title{Der Satz von Hadamard}
%\subtitle{Seminar Klassische Sätze der Differentialgeometrie (WS 21/22)}
\author{Lukas Norkowski}
\date{16. Dezember 2021} 

\newcommand{\RR}{\mathbb{R}}
\newcommand{\ZZ}{\mathbb{Z}}
\newcommand{\DD}{\mathcal{D}}
\newcommand{\MM}{\mathcal{M}}
\newcommand{\KK}{\mathcal{K}}
\newcommand{\HH}{\mathcal{H}}

\renewcommand{\d}{\mathrm{d}}

\hyphenation{Ex-po-nen-tial-ab-bil-dung}
\hyphenation{Über-la-ge-run-gen}

\usefonttheme[onlymath]{serif}
\renewcommand<>{\emph}[1]{{\only#2{\em}#1}}
\theoremstyle{definition}
\newtheorem*{theorem*}{Theorem}
\newtheorem{satz}[theorem]{Theorem}
\newtheorem*{remark}{\translate{Remark}}
\newtheorem*{reminder}{\translate{Reminder}}
\newtheorem{cor}[theorem]{Corollary}
\newtheorem*{cor*}{Corollary}
\newtheorem{proposition}[theorem]{\translate{Proposition}}

\begin{document}

\begin{frame}
	\titlepage
\end{frame}	

\begin{frame}{Gliederung}
\tableofcontents
\end{frame}


\section{Einleitung und Motivation}
\label{Einleitung und Motivation}

\begin{frame}{Das Ziel}
	\begin{satz}[Hadamard]
		\pause
		Es sei $M$ \pause eine einfach zusammenhängende, \pause vollständige \pause Riemannsche Mannigfaltigkeit \pause mit nicht-positiver Schnittkrümmung. \pause Dann ist $M$ diffeomorph zu $\RR^n$, \pause ($n=\dim M$), \pause und für jedes $p\in M$ ist $\exp_p\colon T_pM\to M$ ein Diffeomorphismus.
	\end{satz}
\end{frame}

\begin{frame}{\enquote{Gegenbeispiele}}
	
\end{frame}

\begin{frame}{Noch einmal die Gliederung}
	\tableofcontents
\end{frame}


\section{Konjugierte Punkte und die Exponentialabbildung}
\label{Konjugierte Punkte und die Exponentialabbildung}

\begin{frame}{Konjugierte Punkte}
	\begin{definition}
		\pause
		Sei $\gamma:[0,a]\to M$ eine Geodätische. \pause Ein Punkt $\gamma(t_0)$, \pause $t_0\in (0,a]$, \pause heißt \emph{konjugiert} \pause zu $\gamma(0)$ \pause entlang $\gamma$, \pause falls ein Jacobifeld $J\neq 0$ entlang $\gamma$ existiert \pause mit $J(0)=0 = J(t_0)$.\pause\\
Die Menge $C(p)$ der (ersten) zu $p\in M$ konjugierten Punkte \pause entlang aller Geodätischen ausgehend von $p$ \pause wird \emph{conjugate locus} von $p$ genannt.
	\end{definition}
\end{frame}

\begin{frame}
	\frametitle{Konjugierte Punkte und die Exponential-\\abbildung}
	\begin{Lemma}
		\pause
		Es sei $\gamma\colon [0,a]\to M$ eine Geodätische \pause und setze $p\coloneqq\gamma(0)$. \pause Ein Punkt $\gamma(t_0), \,t_0\in(0,a]$, \pause ist genau dann zu $p$ konjugiert, \pause wenn $\ker (\d \exp_p)_{v_0} \neq \{0\}$, \pause wobei $v_0\coloneqq t_0\gamma'(0) \in T_p M$ ist. 
	\end{Lemma}
	%\begin{remark}
	%	Ein Jacobifeld $J$ entlang einer Geodätischen $\gamma\colon[0,l]\to M$ mit %$J(0)=0$ genügt der Formel
	%	\begin{equation*}
	%		J(t)= \d (\exp_{\gamma(0)})_{t\gamma'(0)}(tJ'(0)),\quad t\in[0,l].
	%	\end{equation*}
	%\end{remark}
\end{frame}

\begin{frame}{Konjugierte Punkte und nicht-positive Schnittkrümmung}
	\begin{Satz}
		\pause
		Es sei $M$ eine vollständige Riemannsche Mannigfaltigkeit \pause mit nicht-positiver Schnittkrümmung. \pause Für jedes $p\in M$ gilt \pause $C(p)=\emptyset$. \pause Insbesondere ist $\exp_p\colon T_pM\to M$ ein lokaler Diffeomorphismus.
	\end{Satz}
\end{frame}


\section{Überlagerungstheorie}
\label{ueberlagerungstheorie}

\begin{frame}{Wiederholung und Neues}
	Im Folgenden sind $B$ und $\widetilde{B}$ nicht-leere topologische Räume.
	\begin{definition}[Überlagerung]
		\pause
		Eine Überlagerung von $B$ ist ein Raum $\widetilde{B}$ \pause zusammen mit einer surjektiven Abbildung $\pi\colon\widetilde{B}\to B$, \pause sodass es für jeden Punkt $p\in B$ eine Umgebung $U\subset B$ gibt, \pause für die $\pi^{-1}(U)$ eine Vereinigung von paarweise disjunkten offenen Mengen $S_i$ ist, \pause sodass $\pi|_{S_i}\colon  S_i\to U$ ein Homöomorphismus ist.
	\end{definition}
	\pause
	\begin{definition}[Pfadhebungseigenschaft]
		\pause
		Es sei $\pi\colon \widetilde{B}\to B$ eine stetige Abbildung. \pause $\pi$ hat die \emph{Pfadhebungseigenschaft}, \pause falls es zu jeder Kurve $\alpha\colon [0,l]\to B$ eine Kurve $\widetilde{\alpha}\colon [0,l]\to\widetilde{B}$ gibt, \pause sodass $\alpha=\pi\circ \widetilde{\alpha}$.
 	\end{definition}
\end{frame}

\begin{frame}{Überlagerungen heben Pfade}
	\begin{Satz}
		\pause
		Sei $\pi\colon \widetilde{B}\to B$ eine Überlagerung, \pause $\alpha\colon [0,l]\to B$ eine Kurve \pause und $\widetilde{p}_0\in\widetilde{B}$ ein Punkt mit $\pi(\widetilde{p}_0)=\alpha(0)\eqqcolon p_0$. \pause Dann gibt es genau eine \emph{Hochhebung} \pause $\widetilde{\alpha}\colon [0,l]\to\widetilde{B}$ von $\alpha$ \pause mit $ \alpha = \pi\circ \widetilde{\alpha}$ \pause und $\widetilde{\alpha}(0)=\widetilde{p}_0$.
	\end{Satz}
\end{frame}

\begin{frame}{Pfadhebungen und lokale Homöomorphismen}
	% Wir wollen aus einem lokalen Diffeomorphismus einen globalen Diffeomorphismus machen.
	\begin{Satz}
		\pause
		Es sei $\pi\colon \widetilde{B}\to B$ eine lokaler Homöomorphismus \pause mit der Pfadhebungseigenschaft. \pause Weiter seien $B$ einfach zusammenhängend \pause und $\widetilde{B}$ wegzusammenhängend. \pause Dann ist $\pi$ ein Homöomorphismus.
	\end{Satz}
	\pause
	\begin{remark}
		\pause
		$\pi$ kann zum Beispiel eine Überlagerungsabbildung sein.
	\end{remark}
\end{frame}


\begin{frame}{\nameref{ueberlagerungstheorie}}
	% Für den folgenden Satz benötigen wir eine Aussage über lokale Homöomorphismen mit der Pfadhochhebungseigenschaft.
	\begin{Satz}
		\pause
		Es sei $\pi\colon \widetilde{B}\to B$ ein lokaler Homöomorphismus \pause mit der Pfadhebungseigenschaft. \pause Weiter sei $B$ lokal einfach zusammenhängend \pause und $\widetilde{B}$ sei lokal wegzusammenhängend. \pause Dann ist $\pi$ eine Überlagerungsabbildung.
	\end{Satz}
	\pause
	\begin{remark}
		\pause
		Jede Mannigfaltigkeit ist sowohl lokal wegzusammenhängend als auch lokal einfach zusammenhängend.
	\end{remark}
\end{frame}

\begin{frame}{Lokale Diffeomorphismen werden zu Überlagerungen}
	\begin{Satz}
		\pause
		Es sei $M$ eine vollständige Riemannsche Mannigfaltigkeit \pause und $f\colon M\to N$ \pause ein surjektiver, \pause lokaler Diffeomorphismus \pause in eine Riemannsche Mannigfaltigkeit $N$. \pause Falls für alle $p\in M$ und alle $v\in T_pM$ gilt, \pause dass $|\d f_p(v)|\geq |v|$, \pause dann ist $f$ eine Überlagerung. 
	\end{Satz}
	%\pause
	%\begin{corollary}
	%	Ist $M$ eine vollständige Riemannsche Mannigfaltigkeit \pause mit %nicht-positiver Schnittkrümmung, \pause $p\in M$, \pause dann ist %$\exp_p\colon T_pM\to M$ eine Überlagerung \pause (mit einer geeigneten %Metrik auf $T_p M$).
	%\end{corollary}
\end{frame}


% Hier sollte alles nur noch kurz eingeblendet werden.
\section{Der Satz von Hadamard}
\label{Der Satz von Hadamard}

\renewcommand{\pause}{}
\begin{frame}{\nameref{Der Satz von Hadamard}}
	\begin{satz}[Hadamard]
		Es sei $M$ \pause eine einfach zusammenhängende, \pause vollständige \pause Riemannsche Mannigfaltigkeit \pause mit nicht-positiver Schnittkrümmung. \pause Dann ist $M$ diffeomorph zu $\RR^n$, \pause ($n=\dim M$), \pause und für jedes $p\in M$ ist $\exp_p\colon T_pM\to M$ ein Diffeomorphismus.
	\end{satz}
\end{frame}

\begin{frame}{Pfadhebungen und lokale Homöomorphismen}
	% Wir wollen aus einem lokalen Diffeomorphismus einen globalen Diffeomorphismus machen.
	\begin{Satz}
		\pause
		Es sei $\pi\colon \widetilde{B}\to B$ eine lokaler Homöomorphismus \pause mit der Pfadhebungseigenschaft. \pause Weiter seien $B$ einfach zusammenhängend \pause und $\widetilde{B}$ wegzusammenhängend. \pause Dann ist $\pi$ ein Homöomorphismus.
	\end{Satz}
	\pause
	\begin{remark}
		\pause
		$\pi$ kann zum Beispiel eine Überlagerungsabbildung sein.
	\end{remark}
\end{frame}

\begin{frame}{Quellen}
	\begin{enumerate}
		\item \emph{Riemannian Geometry}. Manfredo do Carmo. Birkhäuser. 1992.
		\item \emph{Differential Geometry of Curves and Surfaces}. Manfredo do Carmo. Prentice-Hall, Inc. 1976.
	\end{enumerate}
\end{frame}

\end{document}