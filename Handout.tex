\documentclass[a4paper, ngerman]{scrartcl}
\usepackage[T1]{fontenc}
\usepackage[ngerman]{babel}
\usepackage{layout}    %  um die Seitenränder als Bild auszugeben
\usepackage[left=2cm,right=2cm,top=2.5cm,bottom=2.5cm]{geometry}
\usepackage{amsmath}
\usepackage{amsfonts}
\usepackage{amsthm}
\usepackage{amssymb}
\usepackage{booktabs}
\usepackage{mathtools}

\title{Der Satz von Hadamard}
\subtitle{Seminar Klassische Sätze der Differentialgeometrie (WS 21/22)}
\author{Lukas Norkowski}
\date{16. Dezember 2021}

\newcommand{\RR}{\mathbb{R}}
\newcommand{\ZZ}{\mathbb{Z}}
\newcommand{\DD}{\mathcal{D}}
\newcommand{\MM}{\mathcal{M}}
\newcommand{\KK}{\mathcal{K}}
\newcommand{\HH}{\mathcal{H}}

\renewcommand{\d}{\mathrm{d}}

% Disable \pause command
\newcommand{\pause}{}

\newtheoremstyle{mytheorem}% hnamei
{10pt}   % ABOVESPACE
  {\topsep}   % BELOWSPACE
  {\itshape}  % BODYFONT
  {0pt}       % INDENT (empty value is the same as 0pt)
  {\bfseries} % HEADFONT
  {.}         % HEADPUNCT
  {5pt plus 1pt minus 1pt} % HEADSPACE
  {}          % CUSTOM-HEAD-SPEC
	
\theoremstyle{mytheorem}
\newtheorem{theorem}{Theorem}
\newtheorem{satz}[theorem]{Satz}
\newtheorem{hsatz}[theorem]{Hilfssatz}
\newtheorem{corollary}[theorem]{Korollar}
\newtheorem{lemma}[theorem]{Lemma}
\newtheorem{proposition}[theorem]{Proposition}
\theoremstyle{definition}
\newtheorem{definition}[theorem]{Definition}

\pagenumbering{gobble}

\begin{document}
	\maketitle
	\thispagestyle{empty}	

\begin{theorem}[Hadamard]
	Es sei $M$ \pause eine einfach zusammenhängende, \pause vollständige \pause Riemannsche Mannigfaltigkeit \pause mit nicht-positiver Schnittkrümmung. \pause Dann ist $M$ diffeomorph zu $\RR^n$, \pause ($n=\dim M$), \pause und für jedes $p\in M$ ist $\exp_p\colon T_pM\to M$ ein Diffeomorphismus.
\end{theorem}

\subsection*{Konjugierte Punkte und die Exponentialabbildung}

\begin{definition}
	Sei $\gamma:[0,a]\to M$ eine Geodätische. \pause Ein Punkt $\gamma(t_0)$, \pause $t_0\in (0,a]$, \pause heißt \emph{konjugiert} \pause zu $\gamma(0)$ \pause entlang $\gamma$, \pause falls ein Jacobifeld $J\neq 0$ entlang $\gamma$ existiert \pause mit $J(0)=0 = J(t_0)$.\pause\\
Die Menge $C(p)$ der (ersten) zu $p\in M$ konjugierten Punkte \pause entlang aller Geodätischen ausgehend von $p$ \pause wird \emph{conjugate locus} von $p$ genannt.
\end{definition}

\begin{lemma}
	Es sei $\gamma\colon [0,a]\to M$ eine Geodätische \pause und setze $p\coloneqq\gamma(0)$. \pause Ein Punkt $\gamma(t_0), \,t_0\in(0,a]$, \pause ist genau dann zu $p$ konjugiert, \pause wenn $\ker (\d \exp_p)_{v_0} \neq \{0\}$, \pause wobei $v_0\coloneqq t_0\gamma'(0) \in T_p M$ ist. 
\end{lemma}

\begin{satz}
	Es sei $M$ eine vollständige Riemannsche Mannigfaltigkeit \pause mit nicht-positiver Schnittkrümmung. \pause Für jedes $p\in M$ gilt \pause $C(p)=\emptyset$. \pause Insbesondere ist $\exp_p\colon T_pM\to M$ ein lokaler Diffeomorphismus.
\end{satz}

\subsection*{Überlagerungstheorie}

\begin{definition}
	Es sei $\pi\colon \widetilde{B}\to B$ eine stetige Abbildung. \pause $\pi$ hat die \emph{Pfadhebungseigenschaft}, \pause falls es zu jeder Kurve $\alpha\colon [0,l]\to B$ eine Kurve $\widetilde{\alpha}\colon [0,l]\to\widetilde{B}$ gibt, \pause sodass $\alpha=\pi\circ \widetilde{\alpha}$.
\end{definition}

\begin{hsatz}
	Es sei $\pi\colon \widetilde{B}\to B$ ein lokaler Homöomorphismus \pause mit der Pfadhebungseigenschaft. \pause Weiter sei $B$ lokal einfach zusammenhängend \pause und $\widetilde{B}$ sei lokal wegzusammenhängend. \pause Dann ist $\pi$ eine Überlagerungsabbildung.
\end{hsatz}

\begin{hsatz}
	Sei $\pi\colon \widetilde{B}\to B$ eine Überlagerung, \pause $\alpha\colon [0,l]\to B$ eine Kurve \pause und $\widetilde{p}_0\in\widetilde{B}$ ein Punkt mit $\pi(\widetilde{p}_0)=\alpha(0)\eqqcolon p_0$. \pause Dann gibt es genau eine \emph{Hochhebung} \pause $\widetilde{\alpha}\colon [0,l]\to\widetilde{B}$ von $\alpha$ \pause mit $ \alpha = \pi\circ \widetilde{\alpha}$ \pause und $\widetilde{\alpha}(0)=\widetilde{p}_0$.
\end{hsatz}

\begin{hsatz}
	Es sei $\pi\colon \widetilde{B}\to B$ eine lokaler Homöomorphismus \pause mit der Pfadhebungseigenschaft. \pause Weiter seien $B$ einfach zusammenhängend \pause und $\widetilde{B}$ wegzusammenhängend. \pause Dann ist $\pi$ ein Homöomorphismus.
\end{hsatz}

\begin{satz}
	Es sei $M$ eine vollständige Riemannsche Mannigfaltigkeit \pause und $f\colon M\to N$ \pause ein surjektiver, \pause lokaler Diffeomorphismus \pause in eine Riemannsche Mannigfaltigkeit $N$. \pause Falls für alle $p\in M$ und alle $v\in T_pM$ gilt, \pause dass $|\d f_p(v)|\geq |v|$, \pause dann ist $f$ eine Überlagerung. 
\end{satz}

\subsection*{Literatur}
\begin{enumerate}
	\item \emph{Riemannian Geometry}. Manfredo do Carmo. Birkhäuser. 1992.
	\item \emph{Differential Geometry of Curves and Surfaces}. Manfredo do Carmo. Prentice-Hall, Inc. 1976.
\end{enumerate}


\end{document}