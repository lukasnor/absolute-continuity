\documentclass[a4paper, ngerman]{article}
\usepackage[T1]{fontenc}
\usepackage[ngerman]{babel}
\usepackage{layout}    %  um die Seitenränder als Bild auszugeben
\usepackage[left=2cm,right=2cm,top=2.5cm,bottom=2.5cm]{geometry}
\usepackage{amsmath}
\usepackage{amsfonts}
\usepackage{amsthm}
\usepackage{amssymb}
\usepackage{booktabs}
\usepackage{mathtools}
\usepackage{titling}

\pretitle{\begin{center}\LARGE}
\title{Der Hauptsatz der Differential- und Integralrechnung für absolut stetige Funktionen}
\posttitle{\Large \vskip 0.5em Seminar Maßtheorie (SS 22) \par\end{center}}
\author{Lukas Norkowski}
\date{30. Mai 2022}

\newcommand{\RR}{\mathbb{R}}
\newcommand{\ZZ}{\mathbb{Z}}
\newcommand{\DD}{\mathcal{D}}
\newcommand{\MM}{\mathcal{M}}
\newcommand{\KK}{\mathcal{K}}
\newcommand{\HH}{\mathcal{H}}

\renewcommand{\d}{\mathrm{d}}

% Disable \pause command
\newcommand{\pause}{}

\newtheoremstyle{mytheorem}% hnamei
{10pt}   % ABOVESPACE
  {\topsep}   % BELOWSPACE
  {\itshape}  % BODYFONT
  {0pt}       % INDENT (empty value is the same as 0pt)
  {\bfseries} % HEADFONT
  {.}         % HEADPUNCT
  {5pt plus 1pt minus 1pt} % HEADSPACE
  {}          % CUSTOM-HEAD-SPEC
	
\theoremstyle{mytheorem}
\newtheorem{theorem}{Theorem}
\newtheorem{satz}[theorem]{Satz}
\newtheorem{hsatz}[theorem]{Hilfssatz}
\newtheorem{corollary}[theorem]{Korollar}
\newtheorem{lemma}[theorem]{Lemma}
\newtheorem{proposition}[theorem]{Proposition}
\theoremstyle{definition}
\newtheorem{definition}[theorem]{Definition}

\pagenumbering{gobble}

\begin{document}
	\maketitle
	\thispagestyle{empty}	

\begin{theorem}[Fundamentalsatz für stetig differentierbare Funktionen]
	Es sei $f\colon [a,b]\to \RR$ eine differentierbare Funktion mit stetiger Ableitung. Dann gilt\pause
\begin{equation*}
    f(x)-f(a) = \int_a^x f'(t)\d t,\quad x\in[a,b].
\end{equation*}
\end{theorem}

\begin{definition}
	Eine Funktion $f\colon[a,b]\to \CC$ heißt \term{absolut stetig} auf $[a,b]$, falls es zu jedem $\varepsilon>0$ ein $\delta>0$ gibt, so dass für jedes $n\in\NN$ und alle disjunkten Teilintervalle $(a_1,b_1),\dots,(a_n,b_n)$ in $[a,b]$ gilt: 
\begin{equation*}
    \sum_{i=1}^n b_i - a_i <\delta \quad \implies \quad \sum_{i=1}^n \abs{f(b_i) - f(a_i)}<\varepsilon.
\end{equation*}
\end{definition}

\begin{satz}
  Sei $f:[a,b]\to\RR$ stetig und monoton wachsend. Dann sind äquivalent:
\begin{enumerate}
\item $f$ ist auf $[a,b]$ absolut stetig,
\item $f$ bildet Nullmengen auf Nullmengen ab,
\item $f$ ist fast überall differentierbar auf $[a,b]$, $f'\in L^1[a,b]$ und
\begin{equation*}
    f(x)-f(a)=\int_a^x f'(t) \d t, \quad x\in[a,b].
\end{equation*}
\end{enumerate}
\end{satz}

\begin{definition}
  Sei $f\colon [a,b]\to\CC$ eine Funktion. \pause Dann bezeichnet die \term{Totalvariationsfunktion} von $f$ die Funktion $F\colon[a,b]\to [0,\infty]$, definiert durch \pause
\begin{equation*}
    F(x)\coloneqq \sup \sum_{i=1}^N \abs{f(t_i) - f(t_{i-1})}, \quad x\in[a,b],
\end{equation*}
\pause
wobei das Supremum über alle $N\in\NN$ \pause und alle $a =t_0<t_1<\dots<t_n = x$ genommen wird. \pause

$f$ ist von \term{beschränkter Variation}, falls $F(b)<\infty$. \pause $F(b)$ bezeichnet die \term{Totalvariation} von $f$.
\end{definition}

\begin{satz}
  Sei $f:[a,b]\to\CC$ absolut stetig. Dann ist $f$ von beschränkter Variation. Ist $F$ die Totalvariationsfunktion von $f$, so sind die Funktionen $F, F+f$ und $F-f$ alle monoton wachsend und absolut stetig.
\end{satz}

\begin{theorem}
  Sei $f\colon[a,b]\to \CC$ absolut stetig. \pause Dann ist $f$ fast überall differentierbar auf $[a,b]$, \pause $f'\in L^1[a,b]$, \pause und
\begin{equation*}
    f(x) - f(a) = \int_a^x f'(t)\d t, \quad x\in[a,b].
\end{equation*}
\end{theorem}

\subsection*{Literatur}
\begin{enumerate}
	\item \emph{Real And Complex Analysis}. Walter Rudin. Third Edition. McGraw-Hill International Editions. 1987.
\end{enumerate}


\end{document}